\chapter{Technical Considerations}
\label{chapter-technical_considerations}
In the previos chapter we presented different abstract models which can be suitable in the building of a math based search system.
In this section, we present different aspects that need to be considered when mapping from these theoretical models to a concrete implementation.

\section{Digital Representations of Mathematics}
When building any kind of information retrieval system, selecting the right representations of the data can affect the type of algorithms that one can use, the amount of storage, the quality of the data among other factors. Mathematical content is not an exception, and there are available different formats to encode a mathematical expression. One of the most common ways to represent mathematics in scientific documents, which suits CDS's domain, is \LaTeX. 

Another  important standard for the representation of mathematics is MathML, which was first proposed by the World Wide Web Consortium in 1998 as the recommended language to be used in web documents. One of the advantages of MathML over LaTeX, is the fact that it is an application of XML, which makes it very easy process algorithmically. There two different variations of MathML, presentation and content. The first one is oriented to the visual representation of the expression and the second one is more oriented to the semantics of the expression. Its main component is the <apply> tag which represents the function application . 

OpenMath is yet another markup language to represent mathematics and can be used to compliment presentation MathML to add more semantic information. It is mainly used now for defining mathematical concepts into standard content dictionaries which can be imported. MathML is compatible with these dictionaries.

Table \ref{math_languages} presents a comparison on how different languages can express the quadratic equation $x = \frac{-b \pm \sqrt{b^2 - 4ac}}{2a} $


\begin{longtable}{|>{\centering\arraybackslash}m{3.5cm}|>
{\centering\arraybackslash}m{3cm}|>
{\centering\arraybackslash}m{3cm}|>
{\centering\arraybackslash}m{5cm}|
}
\hline 
\textbf{\LaTeX} & 
\textbf{Presentation MathML} & 
\textbf{Content MathML + Presentation Markups} & 
\textbf{OpenMath}  \\ \hline

\scriptsize{\lstinline|$x = \frac{-b \pm \sqrt{b^2 - 4ac}}{2a} $|}
& \scriptsize{\codefont <math display="block"> <mrow>   <mi>x</mi>  <mo>=</mo>  <mfrac> <mrow> <mo>-</mo> <mi>b</mi> <mo>
{\unicodefont \char"00B1} </mo> <msqrt> <msup> <mi>b</mi> <mn>2</mn> </msup> <mo>-</mo> <mn>4</mn> <mo></mo> <mi>a</mi> <mo></mo> <mi>c</mi> </msqrt> </mrow> <mrow> <mn>2</mn> <mo></mo> <mi>a</mi> </mrow> </mfrac> </mrow> </math>}
&
\scriptsize{\codefont <math> <apply> <eq/> <ci>x</ci> <csymbol> <mfrac> <apply> <minus/> <mi>b</mi> {\unicodefont \char"00B1} <mroot> <apply> <minus/> <csymbol> <msup> <mi>b</mi> <mn>2</mn> </msup> </csymbol> <csymbol> <mn>4</mn> <mo>\&times;</mo> <mi>a</mi> <mo>\&times;</mo> <mi>c</mi> </csymbol> </apply> <mrow></mrow> </mroot> </apply> <apply> <times/> <cn>2</cn> <ci>a</ci> </apply> </mfrac> </csymbol> </apply> </math> } 
&
\scriptsize{\codefont 
<OMOBJ> <OMA> <OMS cd = "relation1" name="eq"/> <OMV name="x"/>
    <OMA> <OMS cd="arith1" name="divide"/> <OMA> <OMS cd = "arith1" name="minus"/><OMA> <OMS cd = "arith1" name="times"/>
          <OMA> <OMS cd = "arith1" name="times"/> <OMV name="b"/> <OMV name="+ALE-"/> </OMA> <OMA> <OMS cd="arith1" name="root"/> <OMA> <OMS cd = "arith1" name="minus"/> <OMA> <OMS cd="arith1" name="power"/> <OMV name="b"/> <OMI>2</OMI> </OMA> <OMA> <OMS cd = "arith1" name="times"/> <OMA> <OMS cd = "arith1" name="times"/> <OMI>4</OMI> <OMV name="a"/> </OMA> <OMV name="c"/> </OMA> </OMA> <OMA> <OMI> 2 </OMI> </OMA> </OMA> </OMA> <OMA> <OMS cd = "arith1" name="times"/> <OMI>2</OMI> <OMV name="a"/> </OMA> </OMA> </OMA> </OMOBJ>}
\\
\hline

\caption{Comparison of the different languages for expressing mathematical content} 
\label{math_languages}
\end{longtable}

\LaTeX \ allows the most compact representations of the equations and it is the closest language to the community. However, its processing is not straightforward because a given equation can include multiple commands from different packages. Also there are several ways to represent the same set of symbols, either using its Unicode code-point or through embedded commands. Another difficulty are custom commands that users can define in the preamble of the document. In overall, the interpretation and processing of a mathematical expression in this language would require writing a complete compiler which is not the purpose of this work. MathML offers a better compromise between the length of the representation and the easiness to be processed.

\section{Math Extraction Tools}
One of the main steps in any information retrieval task is the extraction of the searchable content from the collection of documents. At CDS the most common storage format for publications is PDF. A big part of the harvested documents in CDS comes from the ArXiv pre-prints service and from here, the source code (\LaTeX) of the submissions is available for download. Smaller portions of other documents are stored in other formats like Microsoft Word, OpenOffice Documents.
In practice, the main options from where to do the extraction from, are PDF and \LaTeX\  files. Also taking into account the previous discussion on the available format for representing mathematics, we focused our attention in tools that output MathML.  \\
A conscious exploration of available software for extracting mathematical content from this two types of documents allowed us to identify different tools with different approaches. 
\LaTeX\  files are easier to process and all of the tools implement their own compiler that XML + MathML outputs instead of the typical. Since the original math equation is provided in the document, there is relatively little ambiguity in the equation, and the major issues that the systems struggle with is in the coverage of all the possible commands and packages that are available. \\
For extracting equations from PDF files, the process turns into an Optical Character Recognition (OCR). The reconstruction of the equation has to infer the mathematical structure based on physical positioning and size of the symbols, the length of the lines and so on.

Table \ref{math_extraction_tools} presents the main features of the reviewed software. 

\begin{longtable}{|>{\centering\arraybackslash}
m{2.93cm}|>{\centering\arraybackslash}
m{1.8cm}|>{\centering\arraybackslash}
m{1.6cm}|>{\centering\arraybackslash}
m{1.9cm}|>{\centering\arraybackslash}
m{1.6cm}|>{\centering\arraybackslash}
m{1.8cm}|>{\centering\arraybackslash}
m{1.4cm}|
}
\hline 
\small{\textbf{Software}} & \small{\textbf{Type}} & \small{\textbf{Tested version}} & \small{\textbf{Input Format(s)}} & \small{\textbf{Output Format(s)}} & \small{\textbf{Licence Type}} & \small{\textbf{Latest Update}}  \\
\hline
\small{LaTeXML}\cite{latexml} & \small{Standalone Executable} & \small{0.7.9alpha} & \small{\LaTeX} & \small{XML + MathML} & \small{Open Source} & \small{13 Feb 2014}  \\
\hline
\small{TeX4ht}\cite{tex4ht2} & \small{Standalone Executable} & \small{ 2009-01-31-07} & \small{\LaTeX} & \small{XML + MathML} & \small{ Open Source} & \small{11 Jun 2009}  \\
\hline
\small{SnuggleTex}\cite{snuggletex} & \small{Java Library} & \small{1.2.2} & \small{\LaTeX (Small fragments)} & \small{MathML} & \small{Open Source} & \small{24 May 2010}  \\
\hline
\small{LaTeXmathML}\cite{latexmathml} & \small{Javascript Library} & \small{30-October-2007} & \small{\LaTeX} & \small{XML + MathML} & \small{Open Source} & \small{30 Oct 2007}  \\
\hline
\small{Maxtract}\cite{maxtract1}\cite{maxtract2} & \small{Standalone Executable} & \small{v.1752} & \small{PDF} & \small{\LaTeX / Annotated PDF} & \small{Free to download} & \small{15 Nov 2012}  \\
\hline
\small{InftyReader}\cite{infty1}\cite{infty2} & \small{Standalone Executable (On Windows)} & \small{2.9.6.2} & \small{PDF, TIFF, BMP, GIF, PNG} & \small{\LaTeX, XML + MathML} & \small{Commercial} & \small{22 Dec 2013}  \\

\hline
\caption{Comparison of available Mathematical extraction tools}
\label{math_extraction_tools}
\end{longtable}

For extraction from \LaTeX\  files we concluded that the best option is LaTeXML. The tool is still in current development and has lot of support from the academic community. The performance on a small dataset also showed that the results are consistently better than from the other tools. A more thorough test\cite{latexcomparison} also confirms that LaTeXML provides the best results. For PDF files, the trial version of InftyReader worked very well giving better results and processing successfully more files than Maxtract. As said previously, a big portion of the documents stored in CDS, come from pre-printing services like ArXiV were fortunately the source code of an article is available and because of this reason, our base set for this work will consist on extracting equations from \LaTeX\  files. For licensing reasons, for this project we will not integrate our system with InftyReader for the moment and will proceed with LaTeXML as our extraction tool.



\section{Computer Algebra Systems}
Computer Algebra Systems (CASs) are software packages that allow performing symbolic transformations on mathematical objects. They allow to manipulate, simplify and analyse mathematical equations and offer interesting functionalities for a mathematical based search engine. The most important functionalities that we are looking for in such system are simplification and normalization of mathematical expressions and pattern matching. As discussed previously, the language for representing expressions will be MathML, so we also require that such system is able to import this format. Finally, we are also interested in integrating these functionalities in a bigger system, so integration trough APIs was also considered.


\begin{longtable}{|>
{\centering\arraybackslash}m{2cm}|>
{\centering\arraybackslash}m{2.1cm}|>
{\centering\arraybackslash}m{1.9cm}|>
{\centering\arraybackslash}m{1.55cm}|>
{\centering\arraybackslash}m{1.83cm}|>
{\centering\arraybackslash}m{1.95cm}|
}
\hline 
\textbf{\small{CAS}} & 
\textbf{\small{Imports MathML}} & 
\textbf{\small{Expression normalization}} & 
\textbf{\small{Pattern Matching}}  &
\textbf{\small{License Type}}  &
\textbf{\small{Supported APIs}}
\\
\hline
\small{Maple} &  \small{Yes} & \small{Yes} & \small{Yes} & \small{Commercial} & \small{C, Java, VB} \\ \hline
\small{MatLab (Symbolic Math Toolbox)} & \small{Experimental} & \small{Yes} & \small{Yes} & \small{Commercial} & \small{C/C++, Fortran} \\ \hline
\small{Mathematica} & \small{Yes} & \small{Yes} & \small{Yes} & \small{Commercial} & \small{C, Java, .NET} \\ \hline 
\small{Maxima} &  \small{Experimental} & \small{Yes} & \small{Yes} & \small{Open Source} & \small{C++, Java (External Project)} \\ \hline
\small{SymPy} & \small{No} & \small{Yes} & \small{Yes} & \small{Open Source} & \small{Python} \\ \hline

\caption{Comparison of main CAS}
\label{CAS_comparison}
\end{longtable}

Table \ref{CAS_comparison} presents a comparison of the desired features in the main CASs. The systems that better fit our requirements are Maple and Mathematica. For licensing reasons, we will continue our project using Mathematica.




