
\documentclass[a4paper]{article}

\usepackage[bookmarks=false]{hyperref}
\hypersetup{colorlinks,%
	linkcolor=magenta,%
	citecolor=blue}
\hypersetup{pdftitle={Install and use LaTeX for your papers},%
	pdfauthor={Christophe Yamahata},%
	pdfdisplaydoctitle=true}

\title{Using \LaTeX\ to submit manuscripts to peer reviewed journals.}

\author{Christophe Yamahata \footnote{\href{mailto:christophe.yamahata@a3.epfl.ch}{christophe.yamahata@a3.epfl.ch}}}


\date{May 17$^{th}$, 2008}

\begin{document}

\maketitle

\begin{abstract}
Hereafter, I describe all the steps that have to  be followed if you want to 
prepare scientific papers with \LaTeX. This information is for a PC working 
under Windows XP or Vista. I take the example of a submission to 
J. Micromech. Microeng. but you can also find templates for the IEEE J. MEMS
or any \emph{serious} journal.
\end{abstract}


\section*{Basic steps}

\begin{flushleft}
\begin{enumerate}
\item Download the complete MikTeX 2.9 distribution \cite{MikTeX} \\
    (using `Net Installer': \href{http://miktex.org/2.9/setup}{http://miktex.org/2.9/setup}). \\
    It can take time if your network connection is slow\ldots
    
\item Install MikTeX using the same setup executable file.
    This step also takes long time (typically $\sim 30~$min).

\item ONLY ONCE MikTeX has been installed, you can download and install WinShell (\href{http://www.winshell.de/}{http://www.winshell.de/}) \cite{WinShell}.
	\begin{itemize}
    \item  Open `Winshell.exe'
    \item You can check in ``Options'' $>$ ``Programm Calls'' that MikTeX is the distribution used to compile \LaTeX.
	\end{itemize}
	
\item Create a new folder where you will copy  the \LaTeX\ template files 
	\begin{itemize}
	\item First, you can simply open the `.tex' file that I enclose in this e-mail. After conversion with PDFLaTeX, 
	it will look like the PDF file that I also enclose.
	\item Then, try with the template provided by IoP for \emph{J. Micromech. Microeng} \cite{JMM}. \\
    You can download these files and the guidelines here: \\
	\href{ftp://ftp.iop.org/pub/journals/ioplatexguidelines.zip}{ftp://ftp.iop.org/pub/journals/ioplatexguidelines.zip} \\
	\end{itemize}
\end{enumerate}
\end{flushleft}
If you need further information about the \LaTeX\ language, you can check 
``The Not So Short Introduction to \LaTeXe''. You can find it here: \\
	\indent \href{http://tobi.oetiker.ch/lshort/lshort.pdf}{http://tobi.oetiker.ch/lshort/lshort.pdf}


\begin{thebibliography}{10}
\bibitem{MikTeX} MikTeX {\emph{\href{http://www.miktex.org/}{www.miktex.org}  } }
\bibitem{WinShell} WinShell {\emph{\href{http://www.winshell.de/}{www.winshell.de}}}
\bibitem{JMM} J. Micromech. Microeng. {\emph{\href{http://www.iop.org/EJ/journal/JMM/}{www.iop.org/EJ/journal/JMM}} } 
\end{thebibliography}

\end{document}

