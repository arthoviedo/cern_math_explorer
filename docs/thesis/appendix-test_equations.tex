\chapter{Test equations}

\begin{longtable}{|m{0.75cm}|>
{\centering\arraybackslash}m{2.7cm}|>
{\centering\arraybackslash}m{6.5cm}|>
{\centering\arraybackslash}m{5cm}|
}
\hline 
\textbf{Id} & 
\textbf{Description} &
\textbf{MathML code} &
\textbf{Visual Form}


\\
\hline
Eq1 & \scriptsize{ Time-independent Schr\char"00F6dinger equation} & \scriptsize {\codefont <math><mfenced close=")" open="("> <mrow> <mo>-</mo> <mfrac> <msup> <mi>ℏ</mi> <mn>2</mn> </msup> <mrow> <mn>2</mn> <msub> <mi>m</mi> <mi>e</mi> </msub> </mrow> </mfrac> <msup> <mo>$\nabla$</mo> <mn>2</mn> </msup> <mo>+</mo> <mi>V</mi> <mfenced close=")" open="("> <mi>r</mi> </mfenced> </mrow> </mfenced> <mi>ψ</mi> <mfenced close=")" open="("> <mi>r</mi> </mfenced> <mo>=</mo> <mi>E</mi> <mi>ψ</mi> <mfenced close=")" open="("> <mi>r</mi> </mfenced> </math>} & \scriptsize{ $\left (-\frac{\hbar^{2}}{2m_e}\nabla^{2} + V(\mathbf{r})\right )\psi(\mathbf{r}) = E\psi(\mathbf{r})$ }\\ \hline

Eq2 & \scriptsize{Atomic mass in terms of number of proton and number of neutrons} & \scriptsize {\codefont <math> <mi>A</mi> <mo>=</mo> <mi>Z</mi> <mo>+</mo> <mi>N</mi> </math>} & \scriptsize{$A = Z + N$} \\ \hline

Eq3 & \scriptsize{Radiation flux in term of the initial intensity, the absorption coefficient and the thickness of a substance} & \scriptsize {\codefont <math> <mi>I</mi> <mo>=</mo> <msub> <mi>I</mi> <mn>0</mn> </msub><msup> <mi>e</mi> <mrow> <mo>-</mo> <mi>μ</mi> <mi>x</mi> </mrow> </msup> </math>
} & \scriptsize{$I = I_0e^{-\mu x}$ } \\ \hline

Eq4 & \scriptsize{Higgs Boson decaying process into $W^+$ and $W^-$ bosons. This is not exactly a mathematical equation, but a physical process. However it is also expressed in LaTeX and to high energy physics experts, it makes sense to search them.} & \scriptsize {\codefont <math> <mi>H</mi> <mo>→</mo> <msup> <mi>W</mi> <mo>+</mo> </msup> <msup> <mi>W</mi> <mo>-</mo> </msup> </math>} & \scriptsize{$H\rightarrow W^+W^-$ } \\ \hline

Eq5 & \scriptsize{Einstein Field equation} & \scriptsize {\codefont <math> <msub> <mi>G</mi> <mrow> <mi>μ</mi> <mi>ν</mi> </mrow> </msub> <mo>=</mo> <msub> <mi>R</mi> <mrow> <mi>μ</mi> <mi>ν</mi> </mrow> </msub> <mo>-</mo> <mfrac> <mn>1</mn> <mn>2</mn> </mfrac> <mi>R</mi> <msub> <mi>g</mi> <mrow> <mi>μ</mi> <mi>ν</mi> </mrow> </msub> <mo>=</mo> <mfrac> <mrow> <mn>8</mn> <mi>π</mi> <mi>G</mi> </mrow> <msup> <mi>c</mi> <mn>4</mn> </msup> </mfrac> <msub> <mi>T</mi> <mrow> <mi>μ</mi> <mi>ν</mi> </mrow> </msub> </math>
 </math>
 } & \scriptsize{$G_{\mu\nu}\equiv R_{\mu\nu} - {\textstyle 1 \over 2}R\,g_{\mu\nu} = {8 \pi G \over c^4} T_{\mu\nu}$ } \\ \hline

Eq6 & \scriptsize{Luminosity of a star in terms of its temperature and radius} & \scriptsize {\codefont <math><mi>L</mi><mo>=</mo><mn>4</mn><mi>π</mi><msup><mi>R</mi><mn>2</mn></msup><mi>σ</mi><msubsup><mi>T</mi><mrow><mi>e</mi><mi>f</mi><mi>f</mi></mrow><mn>4</mn></msubsup></math>
} & \scriptsize{$ L = 4\piR^2\sigma T^4_{eff} $ } \\ \hline


\caption{Sample equations used during testing}
\label{equations_table}
\end{longtable}
