\chapter{Annexes}
\label{app}


\section{Conception des masques}
\label{masque}



Le tableau \ref{masque} regoupe les dimensions du masque et les dimensions pr�vues .

\begin{table}[!ht]
\renewcommand{\arraystretch}{1}
\centering
\begin{tabular}{l r r }
\toprule
\textbf{}					&\multicolumn{2}{c}{\textbf{Dimensions}} \\ 
\textbf{Param�tres} 					& \textbf{sur masque}	& \textbf{sur wafer} \\
\midrule
Largeur des lames souples			& $3\mu m$ 				& $2.5\mu m$ 	\\
Largeur des lames rigides			& $2.5\mu m$ 			& $2  \mu m$ 	\\
Largeur d'une dent large			& $5 \mu m$				& $4.5 \mu m$	 \\
Largeur d'une dent �troite			& $4 \mu m$				& $3.5 \mu m$	 \\
Gap (rotor/stator)					& $1 \mu m$				& $1.5 \mu m$ 	 \\
\midrule
Largeur des lames du robot v 2.5	& $2.5\mu m$ 				& $2\mu m$ 	\\
Largeur des lames du robot v 2.0	& $2.0\mu m$ 				& $1.5\mu m$ 	\\
Largeur des lames du robot v 1.5	& $1.5\mu m$ 				& $1\mu m$ 	\\
Largeur des lames du robot v 1.0	& $1.0\mu m$ 				& $0.5\mu m$ 	\\
\bottomrule
\end{tabular}
\caption{Dimensions des param�tres sur le masque de la deuxi�me verison et dimensions pr�vues sur wafer}
\label{table:masque}
\end{table}

La fabrication et la mise au point du processus de fabrication a �t� donn� � une personne externe. Le lecteur est invit� � se r�f�rer aux annexes \ref{process flow} et \ref{runcard} pour le processus en lui-m�me.

\section{Constantes physiques et propri�t�s des mat�riaux}
\label{constantes}

Les valeures suivantes ont �t� utilis�es pour les applications num�riques. Afin de s'affranchir des propri�t�s anisotropes du silicium, les valeurs les plus contraignantes de ces propri�t�s ont �t� consid�r�s. Un coefficient de s�curit� de 10 a �t� pris pour les contraintes admissibles maximum.

\begin{table}[!ht]
\renewcommand{\arraystretch}{1}
\centering
\begin{tabular}{lcr@{}l}
\toprule
\textbf{Param�tres} 			& \textbf{Symboles}	& \multicolumn{2}{c}{\textbf{Valeurs}} \\
\midrule
Module de Young					& $E$				& $160$				& $GaOP$ \\
Contrainte admissible 			& $\sigma_{adm}$ 	& $7\times 10^8$ 			& $ Nm^-2$ \\
Permitivit� de l'air 			& $\epsilon$ 		& $8,85\times 10^-12$ 		& $C^2 N^-1m^-2$ \\
\bottomrule
\end{tabular}
\caption{Constantes physiques et propri�t�s des mat�riaux}
\end{table}

%http://fr.wikipedia.org/wiki/Th%C3%A9orie_des_poutres




\section{Contenu du CD-ROM}
\label{cd}


Le CD-ROM contient les fichiers suivants:

\begin{itemize}
\item $Latex$: Rapport du projet de semestre.
\item $Matlab$: Programme de traitement d'image pour mesure dynamique.
\item $Labview$: Programme de commande du micromoteur.
\item $process flow$: Processus de fabrication
\item $runcard$: Protocole du processus de fabrication.
\item $solidworks$: Conception assist�e par ordinateur du moteur.
\item $Cl�win$: Masques sous format .cif
\item $microscopie MEB$: Image prise par Microscopie Electronique � Balayage MEB.
\item $Excel$: Traitement des r�sultats.
\end{itemize}





\section{Processus de fabrication}
\label{process flow}





\section{Protocole du processus de fabrication}
\label{runcard}





\section{Article en cours d'impression: ``Three-Phase Electrostatic Rotary Stepper Micromotor with a Flexural Pivot Bearing''}
\label{annexe_confidential}




\section{Article de la conf�rence Transducers 2009: ``Single mask 3-phase electrostatic rotary stepper micromotor''}
\label{annexe_transducers}






