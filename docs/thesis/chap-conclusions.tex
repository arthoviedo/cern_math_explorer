\chapter{Conclusions}
\label{chap-conclusions}

The area of Mathematical Information Retrieval is still very young and different systems, algorithms, datasets are being developed. Our work, $5e^{x+y}$, proposes a complete MIR system for a concrete digital library as the Cern Document Server. In terms of our proposed goals we can conclude:

\begin{itemize}
\item We explored and identified different tools and approaches to automatically detect and extract mathematical content from digital files. We conclude that Latexml and InftyReader are currently the best suitable set of tools to process Latex and PDF files respectively. We automatized the extraction process and incorporated it into the invenio workflow.
\item Our exploration of the extraction tools, lead us to identify the MathML format as the best format to store and process mathematical content. It is the recommended language for web documents, and the best extraction tools agree in it.
\item Once the mathematical expression is extracted and parsed to a suitable format the next step is identifying suitable features to index. We separated our features into notational and structural ones. For the first category, we developed some issues that have not yet been addressed in previous work and developed heuristics that allow to match more cases by normalizing unicode characters, grouping similar operators and expanding different numeric quantities. For the structural features, we explored the integration of a CAS to perform pattern matching and simplification of the given expression. 
\item We explored different IR frameworks and models to store and retrieve the indexed expressions and found the standard vector space model a suitable one for our needs.  
\item We implemented our system in top of the Solr/Lucene framework. We took advantage of the modular design of Lucene to structure our work in independent classes where each one has a defined role. This structure also allows to easily implement and incorporate further heuristics to handle more cases. Our system is deployed as a Solr plugin making it very easy to be used from an external application. As a final component in our system, we developed the glue code to integrate our search engine with Invenio and developed a simple to use web interface.
\item We performed different set of evaluation on our system to explore the queality of the retrieved results and the performance of our system. Our quality evaluations showed that a combination of notational and structural features allowed to achieved significantly better results. Our performance results showed that for indexing time, the extraction of structural features through the Mathematica CAS, imposes a heavy overhead of more than 10x, but as discussed previously, the extra work is worth with an improved set of results. The extra overhead at querying time is not insignificant but it is still acceptable for today's web usage standards. Our evaluation discarded the tree edit distance because the running time was 200 times than the time needed by a simple angular distance used in standard vector space model implementations, making this approach impractical for a big system.
\item Finally, during this work we identified areas of improvement which are worth looking for. This are presented in the next section.
\end{itemize}

