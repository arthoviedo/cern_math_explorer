\chapter{Future work}
\label{chap-future_work}

Several different work directions were identified during the development of this work. 

\section{Features extraction}
The MathML specification include hundreds of different tags, which were not addressed in our work. They can affect drastically the semantics of a mathematical expression. A simple example is the fact that sometimes to stress that a variable represents a vector, it is standard to use a bold font. The same can be signigied by an arrow on top of a variable. Dozens and even hundreds of these notations contain important information which is not taken into account by our system, since each one of them requires some coding time (TODO: explicit treatment?), therefore handling all of the cases is impractical. Also, there is the limitation of the extracting tools to handle complex constructs like matrices or embedded text.

\section{Data Analysis}
Different algorithms and techniques have been developed to overcome some of the shortcomings of the vector space model. Latent semantic indexing together with singular value decomposition, allow to automatically identify synonyms which can improve the recall of an IR system. During our work, we tested different projects such as Semantic Vectors\cite{semantic_vectors} and \cite{mallet} and Mahout. However, these tools are not yet ready for a big deployment and a considerable amount of time and effort was spent on trying to integrate the formats from the Lucene index to the input format of each tool, as well as debugging and fixing code of some of the tools to make them work. Several times we encountered errors due to the in-memory implementations of the algorithms. Further work on integrating the results of these techniques to the workflow of the system could provide better results. 

In our work, we explored the clustering of the expressions and managed to find a reasonable parameter k (TODO: what is k?). However more insight from this should be developed and the information gained with the clustering step should be integrated in the workflow to improve both the results and the querying time. 

\section{End to end system}
In our implementation, we explored and incorporated \LaTeX extraction to our system. For a full integration on CDS, PDF extraction becomes an important need since most of the records that do (not come from ArXiV, do) not include the source files.
